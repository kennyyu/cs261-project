\documentclass[10pt,twocolumn]{article}

\usepackage{algorithm}
%\usepackage{moreverb}   
%\usepackage{longtable}
\usepackage{fancyhdr}
\usepackage{algorithmic}             
%\usepackage{algorithm}
%\usepackage{array}     
\usepackage{hyperref}
\usepackage{graphicx}
\usepackage{subfigure}
\usepackage{fullpage}
\usepackage{amsmath, amssymb, amsthm}
%\usepackage[framed,numbered,autolinebreaks,useliterate]{mcode}
%\usepackage{mathabx}

\usepackage{float}

\usepackage[margin=1in]{geometry}

\title{{\bf Intrusion Detection using Parzen-Windows on Provenance Graph Statistics}}
\author{
    Kenny Yu\\
    Harvard University\\
    \href{mailto:kennyyu@college.harvard.edu}{\texttt{kennyyu@college.harvard.edu}}
  \and
    R. J. Aquino\\
    Harvard University\\
    \href{mailto:rjaquino@college.harvard.edu}{\texttt{rjaquino@college.harvard.edu}}
}
\date{CS261, Fall 2013}

\begin{document}

\maketitle

%%%%%%%%%%%%%%%%%%%%%%%%%%%%%%%%%%%%%%%%%%%%%%%%%%%%
% ABSTRACT
%

\begin{abstract}
Provenance is data that describes how a digital artifact came to be in its current state. We hypothesize
that intrusions on a system leave behind anomalies in the lineage of digital artifacts.
We present an intrusion detection approach to find these anomalies by analyzing centrality metrics
on provenance graphs. We use a Parzen-Window approach (TODO CITE) on various provenance graph centrality metrics (TODO CITE)
to determine probability density estimates of normal behavior, and we use these density estimates to 
determine if an intrusion occurred. We used this approach to analyze {\em user-to-remote} (u2r) intrusions and 
{\em remote-to-local} (r2l) intrusions (TODO: include r2l?) from the 1998 DARPA Intrusion Detection data sets (TODO CITE) and 
achieved up to *TODO true positive rate for intrusions* accuracy in detecting
intrusions with only *TODO false positive rate for intrusions* accuracy. We also present future work
to extend our intrusion model to an online intrusion detection system.

\end{abstract}

%%%%%%%%%%%%%%%%%%%%%%%%%%%%%%%%%%%%%%%%%%%%%%%%%%%%
% INTRODUCTION
%

\section{Introduction}

For as long as systems have existed, there have been malicious users that attempt to exploit vulnerabilities
in systems to gain unintended privileged access. As a result, system designers and 
administrators place a large effort in securing systems and preventing intrusions. However, intrusions
inevitably occur because of bugs or flaws within the system, and as a result, system administrators want to
have some automatic way of detecting these intrusions. 

Intrusions often leave behind digital artifacts, e.g., unusual output or log files, a process executed
with an unusual arguments or environment, unusual system call activity by a process. In addition to
unusual digital artifacts, we hypothesize that
intrusions also intrusions leave behind abnormalities
in the {\em provenance} of these digital artifacts, the lineage of how these digital artifacts were created.

Provenance is data that describes how digital artifacts
came to be in their current state. Provenance data is typically structured as a directed acyclic graph with
typed nodes and typed edges.
Nodes typically include processes and files, and edges are directed from nodes to their dependencies. For example,
process nodes have edges directed towards input file nodes and to the parent process's node, and
file nodes have edges directed towards previous versions of the file and to the nodes of the
processes that modified the file.

Existing intrusion detection systems (IDS) use provenance data only to a limited extent.
??? person analyzes
basic statistics on provenance graphs (e.g., number of nodes, number of edges, ... TODO) in order to facilitate
easier manual intrusion detection by a human (CITE WORK). Their work, however, does not present a way of using provenance
graph to automatically determine intrusions. Other systems use provenance data to determine the scope of an
intrusion. For example, Backtracker (CITE WORK) uses provenance graphs to build causality graphs of intrusions: once the
system has
determined an intrusion has occurred, the system follow the causality graph backwards determine the original source of the attack, and 
then it follow the causality graph forwards to determine all the objects tainted by the attack. However, the authors 
do not present a way of using provenance data to automatically detect intrusions.


>>> INSERT PROVENANCE GRAPH SMALL EXAMPLE HERE SHOWING DATA FLOW

In this paper, we present an approach to use provenance data to automatically detect intrusions. Intrusion
detection can be framed as a {\em novelty detection} problem, in which one attempts to decide whether
an unknown test pattern is produced by an underlying distribution corresponding to a training set
of normal patterns (CITE WORK). However, ???authors note that in the case of intrusion detection, novel
or abnormal patterns are typically difficult to obtain, and as a result, they present a {\em Parzen-Window} approach
for non-parametric density estimation. Using this technique, they obtain a high degree of success
in identifying various intrusion types (CITE WORK). Because obtaining abnormal patterns is difficult, 
we borrow their Parzen-Window approach to build models of normal behavior using 
various provenance graph statistics and graph centrality metrics, and we evaluate the success of this technique
on {\em user-to-local} (u2l) intrusions (intrusions that provide unauthorized access to local superuser (root) privileges),
and {\em remote-to-local} (r2l) intrusions (intrusions that provide unauthorized access from a remote machine).

The main contributions of this paper are the following:
\begin{enumerate}
\item Discuss which provenance graph statistics we chose to use, how we chose them, and why we chose them.
\item Analyze the Parzen-Window technique on a real intrusion detection data set and evaluate its performance.
\item Discuss the limitations of the approach and present future work to transform the technique into an online intrusion detection system.
\end{enumerate}


%%%%%%%%%%%%%%%%%%%%%%%%%%%%%%%%%%%%%%%%%%%%%%%%%%%%
% RELATED WORK
%

\section{Related Work}

\subsection{Existing IDSs}

inspiration
- novelty detection problem with intrusion detection
- somayaji use sequences of system calls
- system calls and BSM data can be used to build provenance graphs; we believe that
provenance graphs provide more information than somayaji and can help detect
intrusions with better results

use provenance to automatically detect intrusions
- existing machine learning techniques on KDD data set: use features of an intrusion (number of data)
- but not actually using graph structure
- parzen window--allow us to build a profile of normal with lack of abnormal examples

\subsection{Provenance Graph Features}

how we choose features to look at
- basic statistics for manual detection
- graph centrality and clustering can help us identify the different kinds of tasks running
- use these metrics for kernel estimation and 

\subsection{Types of Intrusions}

we limit our work to these intrusions
- explain various kinds
- why just user 2 local



%%%%%%%%%%%%%%%%%%%%%%%%%%%%%%%%%%%%%%%%%%%%%%%%%%%%
% DESIGN
%

\section{Design and Implementation}

\subsection{Selecting Metrics}

talk about PASS and SPADE here.

put that table here

%%%%%%%%%%%%%%%%%%%%%%%%%%%%%%%%%%%%%%%%%%%%%%%%%%%%
% EVALUATION
%

\section{Evaluation}

\subsection{Experimental Setup}

\subsection{Results}

\subsection{Discussion}


%%%%%%%%%%%%%%%%%%%%%%%%%%%%%%%%%%%%%%%%%%%%%%%%%%%%
% CONCLUSION
%

\section{Conclusion}

\section{Limitations \& Future Work}


%%%%%%%%%%%%%%%%%%%%%%%%%%%%%%%%%%%%%%%%%%%%%%%%%%%%
% BIBLIOGRAPHY
%

\begin{thebibliography}{99}

\bibitem{fuzzy}
\textsc{Cao, D., Qiu, M., Chen, Z., Hu, F., Zhu, Y., and Wang, B.} Intelligent Fuzzy Anomaly Detection of Malicious Software. In {\em Internal Journal of Advanced Intelligence}, vol. 4, no. 1, pp 69-86 (December 2012).

\bibitem{somayaji-recent}
\textsc{Inoue, H. and Somayaji, A.} Lookahead Pairs and Full Sequences: A Tale of Two Anomaly Detection Methods. In {\em 2nd Annual Symposium on Information Assurance} (June 2007). 

\bibitem{backtracker}
\textsc{King, S. T. and Chen, P. M.} Backtracking Intrusions. In {\em SOSP'03 Proceedings of the nineteenth ACM symposium on Operating systems principles} (December 2003).

\bibitem{multihost}
\textsc{King, S. T., Mao Z. M., Lucchetti, D. G., and Chen, P. M.} Enriching intrusion alerts through multi-host causality. In {\em Proceedings of the 2005 Network and Distributed System Security Symposium} (February 2005).

\bibitem{fileprefetch}
\textsc{Lei, H. and Duchamp, D.} An Analytical Approach to File Prefetching. In {\em Proceedings of the USENIX 1997 Annual Technical Conference} (January 1997).

\bibitem{clustering}
\textsc{Macko, P., Margo, D., Seltzer, M.} Local Clustering in Provenance Graphs (Extended Version). In {\em Proceedings of the 22nd ACM international conference on Conference on information \& knowledge management} (August 2013).

\bibitem{orbiter}
\textsc{Macko, P. and Seltzer, M.} Provenance Map Orbiter: Interactive Exploration of Large Provenance Graphs. In {\em TaPP'11 Proceedings of the 2nd conference on Theory and practice of provenance} (June 2011).

\bibitem{fileattributes}
\textsc{Margo, D., and Smogor, R.} Using Provenance to Extract Semantic File Attributes. In {\em TaPP'10 Proceedings of the 2nd conference on Theory and practice of provenance} (February 2010).

\bibitem{passv2}
\textsc{Muniswamy-Reddy, K., Braun, U., Holland, D. A., Macko, P., Maclean, D., Margo, D., Seltzer, M., and Smogor, R.} Layering in Provenance Systems. In {\em Proceedings of the 2009 USENIX Annual Technical Conference} (June 2009).

\bibitem{pass}
\textsc{Muniswamy-Reddy, K., Holland, D. A., Braun, U., and Seltzer, M.} Provenance-Aware Storage Systems. In {\em Proceedings of the 2006 USENIX Annual Technical Conference} (June 2006).

\bibitem{exploitdb}
\textsc{Offensive Security, Inc.} The Exploit Database. {\tt http://www.exploit-db.com}.

\bibitem{metasploit}
\textsc{Rapid 7 Inc.} Metasploit Framework. {\tt http://www.metasploit.com}.

\bibitem{somayaji}
\textsc{Somayaji, A. and Forrest, S.} Automated Response Using System-Call Delays. In {\em Proceedings of the 2000 USENIX Annual Technical Conference} (August 2000).

\bibitem{correlated-anomalies}
\textsc{Tariq, D., Baig, B., Gehani, A., Mahmood, S., Tahir, R., Aqil, A., and Zaffar, F.} Identifying the provenance of correlated anomalies. In {\em SAC'11 Proceedings of the 2011 ACM Symposium on Applied Computing} (March 2011).

\end{thebibliography}

\end{document}
